\pagestyle{fancy}

\fancyhf{}

\fancyhead[C]{Conclusione}
\fancyfoot[C]{\thepage}

\fancypagestyle{plain}{
    \fancyhf{}
    \fancyfoot[C]{\thepage}
}

\chapter{Conclusione}

\large

La presente tesi ha esplorato l'efficacia di diversi metodi di machine learning per il rilevamento dello stress negli ambienti di lavoro d'ufficio. I risultati ottenuti hanno fornito importanti intuizioni su come lo stress impatti i dipendenti in diversi contesti lavorativi e hanno evidenziato l'efficacia di specifici strumenti di rilevamento.

\bigskip

Gli autori del dataset SWELL-KW, attraverso l'uso di questionari, monitoraggio fisiologico e interviste, hanno osservato che i livelli di stress possono variare notevolmente in base a fattori come il carico di lavoro, il clima organizzativo e le relazioni interpersonali sul posto di lavoro. Gli strumenti di rilevamento utilizzati si sono dimostrati validi per identificare i segnali di stress e per aiutare le organizzazioni a sviluppare interventi mirati.

\bigskip

In particolare, i test hanno confermato che l'apprendimento supervisionato è in grado di effettuare previsioni più precise. Allo stesso tempo, i modelli non supervisionati non mostrano prestazioni particolarmente scadenti, ma sembrano richiedere dati aggiuntivi per creare cluster più compatti. Questo risultato riafferma quanto evidenziato nel capitolo sullo stato dell'arte: i modelli di deep learning, come le reti neurali, e quelli basati su architetture con nodi e connessioni, si dimostrano estremamente efficaci.

\bigskip

Questo paragrafo, è dedicato a eventuali progetti futuri, in quanto questo progetto potrebbe essere approfondito. Pur essendo il dataset pubblicamente accessibile su Kaggle, è importante notare che i segnali provenienti dall'elettrocardiogramma sono già stati preprocessati e convertiti in dati numerici, così escludendo al pubblico una fase significativa di ricerca. Un secondo punto critico riguarda la grande quantità di dati presente nel dataset, che può portare alcuni modelli a soffrire di overtraining. Questo aspetto è rilevante poiché molti notebook online trascurano tale problematica, compromettendo la realistica interpretazione dei risultati dei loro studi. Infine, sarebbe interessante esplorare l'approccio del deep learning per valutare le prestazioni dei modelli e la loro capacità di apprendere e predire in modo accurato.

\bigskip

In conclusione, questo studio ha evidenziato l'importanza di un approccio basato sui dati per affrontare il problema dello stress negli ambienti di lavoro. Continuando a sviluppare e implementare strategie efficaci, è possibile creare ambienti lavorativi più sani e produttivi.