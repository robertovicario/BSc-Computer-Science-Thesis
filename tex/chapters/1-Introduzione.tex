\pagestyle{fancy}

\fancyhf{}

\fancyhead[C]{Introduzione}
\fancyfoot[C]{\thepage}

\fancypagestyle{plain}{
    \fancyhf{}
    \fancyfoot[C]{\thepage}
}

\chapter{Introduzione}

\large

Il rilevamento dello stress negli ambienti di lavoro, rappresenta un elemento innovativo per promuovere la salute e il benessere dei dipendenti, soprattutto negli embienti di lavoro d'ufficio. L'utilizzo del machine learning in questo caso offre un approccio efficace per identificare segnali precoci di stress, consentendo alle organizzazioni di intervenire tempestivamente per mitigare i rischi associati. In particolare, è di nostro interesse esaminare se i risultati ottenuti attraverso metodi di apprendimento non supervisionato possano essere paragonabili a quelli derivanti da approcci supervisionati, come evidenziato nella ricerca \textit{"Exploring Unsupervised Machine Learning Classification Methods for Physiological Stress Detection"} di Iqbal et al. (2022) \cite{iqbal2022exploring}.

\bigskip

Il dataset SWELL-KW, è stata la risorsa fondamentale da cui prelevare i dati per allenare i modelli di intelligenza artificiale a intercettare lo stress nei lavoratori dipendenti. All'interno del capitolo, viene esplorato il processo di raccolta dei dati, delineando le metodologie utilizzate per garantire la qualità del dataset. Successivamente, vengono analizzate le features estratte dai dati e le relative labels associate.

\bigskip

In seguito, si procede con l'analisi dei segnali biomedici, attribuendo loro un'importanza fondamentale nel contesto della ricerca. Viene esposta la distinzione tra segnali fisici e fisiologici, con un focus approfondito sugli specifici segnali impiegati. Particolarmente rilevante è la frequenza cardiaca, rappresentata tramite l'elettrocardiogramma (ECG), che assume un ruolo fondamentale nelle indagini sul rilevamento dello stress. L'esplorazione di questi temi costituisce una base essenziale per comprendere le principali metodologie di machine learning nell'ambito della variabilità della frequenza cardiaca (HRV).

\bigskip

Successivamente, vengono presentate le ricerche allo stato dell'arte sul rilevamento dello stress utilizzando metodi di machine learning. Le metodologie e i risultati esposti nel capitolo quattro rivestono un'importanza cruciale nel tracciare lo stato attuale della ricerca e nel fornire un contesto significativo per future indagini. Questi elementi sono particolarmente rilevanti per la definizione delle metodologie e la comparazione dei risultati ottenuti in questo progetto.

\bigskip

Infine, si espone la metodologia adottata, a partire dalle tecnologie impiegate nello sviluppo del progetto fino ai metodi di machine learning utilizzati per apprendere e valutare i risultati. Nei capitoli cinque e sei, il progetto è dettagliato sia dal punto di vista teorico che pratico, approfondendo l'analisi che ha condotto all'implementazione. In conclusione, vengono presentati e confrontati i risultati ottenuti dai modelli generati tramite queste metodologie con quelli allo stato dell'arte. Infine, si delineano le linee guida per future ricerche sul rilevamento dello stress mediante l'utilizzo del machine learning, sfruttando il dataset SWELL-KW per analizzare lo stress negli ambienti di lavoro d'ufficio.